\section{RealNVP Transformations}

\cite{RealNVP} introduced a very simple invertible function of the form:

\begin{nalign}
\left\{ 
    \begin{array}{ll}
    t(\boldz)_{1:d} &= \boldz_{1:d}
    \\
    t(\boldz)_{d+1:K} &= \boldz_{d+1:K}\odot \exp\left(s(\boldz_{1:d})\right) + a(\boldz_{1:d})
    \end{array}
\right.
\end{nalign}

    The inverse can be trivially obtained as:

\begin{nalign}
\left\{
    \begin{array}{ll}
    \boldz_{1:d} & = t(\boldz)_{1:d}\\
    \boldz_{d+1:K} &= \left( t(\boldz)_{1:d} - a(\boldz_{1:d}) \right) \underbrace{\oslash \exp(s(t(\boldz_{1:d})))}_{\odot \exp(-s(t(\boldz_{1:d})))}
    \end{array}
\right.
\end{nalign}

$s(\cdot)$ can be any dimensionality-preserving nonlinear function, such as a neural network with nonlinear activations. $a(\cdot)$ is an affine transformation.
In this work's implementation $d$ is set $d = K/2$. 

The main advantage of using such transformations is that the Jacobian matrix is triangular,
hence its determinant is obtained 
from the diagonal, culminating with the form
    $\exp\left(\sum_j s(\boldz_{1:d})_j \right)$ 

\subsection{Masking}

To ease the implementation, the selection of the first and second parts of $\boldz$
have been implemented with random hyperparameter masks. These masks are unique for each transformation
step, and are computed as follows:

\begin{algorithm}[H]
\caption{Half-full random masks for RealNVP transformations}
\begin{algorithmic}[1]

\REQUIRE ~~\\
(1) Latent dimensionality $K$ \\
(2) Number of transformation steps $k$
\ENSURE~~\\
(1) Random masks $\boldm_1 \ldots \boldm_k$ which have half of their elements set at $1$
`
\item[]
\FOR{$i \in \{1 \ldots k\}$}
\STATE $(\mathbf{a})_j \leftarrow \left\{\begin{array}{ll} 1 & j < K/2 \\ 0 & K/2 \leq i < K\end{array}\right.$
\STATE $\boldm_i = \mathrm{shuffle}(\mathbf{a})$
\ENDFOR
\RETURN $\boldm_1 \ldots \boldm_k$
\end{algorithmic}
\end{algorithm}
                                         
The invertible function, for a transformation step $i$, is hence implemented as:

\begin{nalign}
    \boldz_{i+1} &= \boldz_i\odot\boldm_i + (1-\boldm_i)\odot\left[\boldz_i \odot \exp\left(s(\boldm_i\odot\boldz)\right) + a(\boldm_i\odot\boldz)\right]
\end{nalign}

The masks are computed before training and are left unchanged, as they
should be considered as hyperparameters.
