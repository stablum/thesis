\section{Free bits}

\subsection{Diagnosing the collapse of the posterior approximation to the prior distribution}

It was observed\cite{Kingma2017}\cite{1611.02731} (FIXME: cite Bowman et al., 2015 and Kaae Sønderby et al., 2016 )
that in the initial phases of training, due to weakness of the term $\pxcond$ the term $\kl{\pzonly}{\qzcond}$ 
induces $\qzcond$ to collapse to the prior $\pzonly$.

Such phenomenon can be diagnosed by looking at the individual Kullback-Leibler divergences
at each latent dimension.

In the case of the diagonal-covariance Multivariate Normal posterior approximation
(the regular VAE), it's easy to find these individual KL values, as it can be analytically determined as:

\begin{nalign}
KL = \log \frac{\sigma_2}{\sigma_1} + \frac{\sigma_1^2 + ( \mu_1 - \mu_2 )^2}{2\sigma_2^2}-\frac{1}{2}
\end{nalign}

As the prior $\pzi$ is $\normal{\boldzi}{\mathbf{0}}{\mathbf{I}}$, the KL per-dimension is reduced to:

\begin{nalign}
KL = -\log \sigma_1 + \frac{\sigma_1^2 + \mu_1^2}{2}-\frac{1}{2}
\end{nalign}

 The $\kl{\qzcond}{\pzonly}$ term of the $\elbox$, if seen in the context of averaging within a minibatch $\mathcal{M}$, as in
 $\expectxM{\kl{\qzcond}{\pzonly}}$,
 can be interpreted as an approximation to a mutual information term $\mutinf{\boldz}{\boldx}$.
 The implied minimization of the mutual information during optimization of the ELBO forces a high dependence of the $\boldx$ datapoints to the prior $\qzonly$,
 leading to over-regularization of $\qzcond$.

 In order to prevent the collapse of the posterior approximation to the prior, the gradients of the mutual information term can be zeroed by setting a lower-bound
 value to the \emph{nats} expressed from that term, as in:
\begin{nalign}
     \max\left[\lambda,\expectxM{\kl{\qzcond}{\pzonly}}\right]
\end{nalign}

FIXME: why not try a smooth function instead of max?
