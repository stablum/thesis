\section{Choice of hyperparameters}

The search for an acceptable model via hyperparameter search was a very long process.
Since most experiments took about 8 days to complete, choosing the right hyperparameter
settings took many months.

At the end of this process it was understood that most hyperparameters were located
on a specific trade-off minimum in a convex curve of validation error.
For example, using minibatch of size 1 gave
often unsatisfactory resuls. Better results were obtained with a minibatch of size 64,
but increasing that value, for example to 128 or 256 gave worse results.

Other hyperparameters that were problematic to set were those dedicated to 
L2 regularization of the network weights. A good balance was found by using 100 or 200 
as L2 regularization coefficient.

The learning rate was also located as a minimum in a convex curve.
High learning rates might initially progress faster but may "jump over" good minima,
while lower learning rates might converge to a better minima but take a larger amount of
epochs. These effects might be mitigated by the use of moment-based descent algorithm,
such as
Adam \cite{KingmaB14} and by the use of 
the learning rate annealing described in section \cite{annealing}, with
parameter $T$ set at $10$, meaning that the initial halving of the learning rate
happens after the first $10$ epochs (further decay is much slower)
. A good initial learning rate was found being $2e-6$.

The ideal number of epochs would have been $1000$ but unfortunately reaching
this target was highly inpractical by the sheer amount of time that the learning
required. This fact was aggravated by the limit on the number of concurrent
jobs that was imposed by the distributed supercomputer DAS4 administration,
which was about 10-20 long-running jobs at the same time.
For these reasons many reported experiments have been trained for a lower number of
epochs.

The depth of the network was finally chosen to be 1 hidden layer,
as it eases the creation of useful intermediate-level
representation values, as opposed to not using hidden layers at all.

Latent dimensionalities explored were 5, 250 and 500. 



