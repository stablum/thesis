\section{Equivalences between AutoRec and VaeRec models}

In order to perform an adequate comparison between the AutoRec and VaeRec
models it's important to establish if there are any available equivalences.
In other words, it is interesting to see if a specific choice of hyperparameters
of the VaeRec leads to a model that is similar to the VaeRec both in its definition
and its performance.

Luckily such a model can be found in the VaeRec by setting the KL coefficient to 0.
This way that extra regularization term is absent and the VaeRec model becomes analogous
to the AutoRec model. An hypothesis can be formulated that in such similar models
the performances will be similar as well.

Experimental results confirm the hypothesis:

\begin{table}[H]
\centering
\begin{tabular}{c|c|c|c|r|r}
\thead{Minibatch \\size }& 
\thead{hid.layer \\ width }& 
\thead{num. hidden \\layers } &
\thead{latent z \\ dimensionality} & 
\thead{AutoRec (RProp) \\ testing RMSE }&
\thead{VaeRec (Adam) \\ testing RMSE }
\\
\hline
64 & 1000 & 1 & 250 & 
% harvest_autorec_20180625_122221
0.8700
% harvest_autorec_20180723_114300 (pseudo_linear, and adam)
WAITING
& 
% harvest_vaerec_20180608_231009
0.8335
% harvest_vaerec_20180723_110214 (pseudo_linear)
WAITING
\\
64 & 1000 & 2 & 250 & 
% harvest_autorec_20180427_022040
0.8341 
% harvest_autorec_20180723_120629 (pseudo_linear, and adam)
WAITING
% harvest_autorec_20180723_121327 (adam)
WAITING
& 
% harvest_vaerec_20180608_224259
0.8365 
% harvest_vaerec_20180723_115919 (pseudo_linear)
WAITING
\\
64 & 1000 & 1 & 500 & TODO  & TODO  \\
64 & 1000 & 2 & 500 & TODO  & TODO  \\
128 & 1000 & 1 & 250 & TODO  & TODO  \\
128 & 1000 & 2 & 250 & TODO  & TODO  \\
128 & 1000 & 1 & 500 & TODO  & TODO  \\
128 & 1000 & 2 & 500 & TODO  & TODO  \\
\end{tabular}
\caption{Comparison of similar VaeRec and AutoRec models}
\end{table}

The testing error achieved by both AutoRec and VaeRec models
are very similar under similar hyperparameter settings.
