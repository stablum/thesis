\section{Equivalences between AutoRec and VaeRec models}

In order to perform an adequate comparison between the AutoRec and VaeRec
models it's important to establish if there are any available equivalences.
In other words, it is interesting to see if a specific choice of hyperparameters
of the VaeRec leads to a model that is similar to the VaeRec both in its definition
and its performance.

Luckily such a model can be found in the VaeRec by setting the KL coefficient to 0.
This way that extra regularization term is absent and the VaeRec model becomes analogous
to the AutoRec model.

The ELBO function per-datapoint of a VAE 
with posterior distribution being diagonal-covariance gaussians,
without the KL divergence becomes:

\begin{nalign}
\elboxi
&= \expectqphi{\log \pxicond}\\
\end{nalign}

If $\pxicond$ has a spherical gaussian ($I$ covariance matrix) form,
then this objective, which comprises only the likelihood term,
becomes very similar to the reconstruction error
of a regular autoencoder, but differs for the fact that $\boldzi$
is stochastic and drawn from a distribution determined by the encoder.
Since the KL term is absent, $\qphizcondi$, unregolarized,
will tend to collapse to distributions that are centered in specific $\boldmu$'s in latent
space but have $\boldsigma$'s that tend to 0, hence with
random sample from $\qphizcondi$ being always $\boldmu$.

Hence, an hypothesis can be formulated about the similarity of VaeRec without KL
and AutoRec:

Experimental results confirm the hypothesis by showing similarity of testing error:

\begin{table}[H]
\centering
\begin{tabular}{c|c|c|c|r|r}
\thead{Minibatch \\size }& 
\thead{hid.layer \\ width }& 
\thead{num. hidden \\layers } &
\thead{latent z \\ dimensionality} & 
\thead{AutoRec (RProp) \\ testing RMSE }&
\thead{VaeRec (Adam) \\ testing RMSE }
\\
\hline
64 & 1000 & 1 & 250 & 
% harvest_autorec_20180625_122221 (adam)
0.8700
% harvest_autorec_20180723_114300 (pseudo_linear, and adam)
WAITING
& 
% harvest_vaerec_20180608_231009
0.8335
% harvest_vaerec_20180723_110214 (pseudo_linear)
WAITING
\\
64 & 1000 & 2 & 250 & 
% harvest_autorec_20180427_022040
0.8341 
% harvest_autorec_20180723_120629 (pseudo_linear, and adam)
0.8500
% harvest_autorec_20180723_122618/ (adam)
0.8526
& 
% harvest_vaerec_20180608_224259
0.8365 
% harvest_vaerec_20180723_115919 (pseudo_linear)
0.8495
\\
64 & 1000 & 1 & 500 & 
% NO (rprop)
% NO (adam)
% harvest_autorec_20180725_005108/ (pseudo_linear, and adam)
WAITING
& 
% NO (no gradient hacks)
% harvest_vaerec_20180725_004522/ (pseudo_linear)
0.8767
\\
64 & 1000 & 2 & 500 & 
% NO (rprop)
% NO (adam)
% harvest_autorec_20180725_005647/ (pseudo_linear, and adam)
0.8696
& 
% NO (no gradient hacks)
% harvest_vaerec_20180725_010335/ (pseudo_linear)
0.8511
\end{tabular}
\caption{Comparison of similar VaeRec and AutoRec models}
\end{table}

The testing error achieved by both AutoRec and VaeRec models
are very similar under similar hyperparameter settings.
