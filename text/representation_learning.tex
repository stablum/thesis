\section{Representation Learning}
Representation Learning (RL) is a developing branch of Machine Learning that has one of its
focuses on extracting representations codes $$\mathbf{Z} = \{\mathbf{z}_i\}_{i=1}^N$$ from the datapoints in a dataset $$\mathbf{X} = \{\mathrm{x}_i\}_{i=1}^N$$.
It is usually desirable for these representations to be characterized by properties such as low-dimensionality, clusterability, increased linear separability (especially when used for further classification tasks) and intuitive "semantic" explainability of the dimensions of the learned manifold.

One common attempt to achieve such properties is the use of Principal Component Analysis (PCA) , which transforms the original features of the raw input into a set of uncorrelated variables.
The main drawback of PCA is the assumption that the explaining dimensions are linearly related to the directions of maximum variance in the data.
This assumption is not true for most complex datasets, in which the original features of a dataset might actually be the result of arbitrarly complicated nonlinear unknown functions.

For this reason alternative approaches to RL are being employed, such as Autoencoders (AE) as specific forms of Artificial Neural Networks (ANN).
AEs have typically a "diabolo" shape, as an arbitrarly highly dimensional input is progressively being reduced to lower dimensionalities over layers of progressively shrinking sizes.
This part of the neural network is an encoder, as its purpose is to generate low-dimensional compressed codes from a high-dimensional input.
The last layer of the encoder is the smallest layer of the network, hence called bottleneck.
To the bottleneck is attached a decoder network with layers of progressively increasing size.
The last layer of the decoder is matching the dimensionality of the input layer.
The learning of the network's parameters is performed by minimizing an objective function containing the error between the reconstruction output of the network and its input.
This objective function is then minimized via Gradient Descent and its variants.
