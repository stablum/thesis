\subsection{RPROP update algorithm}

\emph{RPROP}\cite{rprop} is a gradient-based parameter update schema that does not take into account the magnitudes of the gradients, but only their sign.

The idea is simple: if the gradient keep pointing towards the same (either positive or negative) direction,
then the parameter-specific update delta needs to be increased, otherwise, in case
the gradient for a parameter keeps changing sign, then the update delta needs to decrease,
in order to fine-tune the parameter to its optimum.

The change of variation is detected by the product of the gradient of an parameter $w_i$
calculated to minimize an objective function $J$
parameter a time step $t$ by the gradient of that same parameter at the previous time step $t-1$:

\begin{nalign}
p &= \left(\frac{\partial J}{\partial w_i}\right)^{(t-1)}
* \left(\frac{\partial J}{\partial w_i}\right)^{(t)}
\end{nalign}
       
The sign of $p$ determines the increase or decrease of the parameter-specific delta $\Delta_i$:
\begin{nalign}
\Delta_i^{(t)} =
\left\{
\begin{array}{lll}
 \min \{ \eta^+ * \Delta_i^{(t-1)} , \Delta_{\mathrm{max}} \} & \mathrm{if} & p > 0\\
 \max \{ \eta^- * \Delta_i^{(t-1)} , \Delta_{\mathrm{min}} \} & \mathrm{if} & p < 0\\
 \Delta_i^{(t-1)} & \mathrm{if} & p = 0
\end{array}
\right. 
\end{nalign}

Where $ 0 < \eta^- < 1 < \eta^+ $.

Typical parameter settings are $\eta^- = 0.5$, $\eta^+ = 1.2$, $\Delta_{\mathrm{min}} = 1e^{-6}$,
$\Delta_{\mathrm{max}} = 50.0$ and initial delta values $\Delta_0 = 0.1$.

\emph{RPROP} has been used specifically for \emph{AutoRec},
    as suggested in the original paper \cite{Sedhain2015}.
