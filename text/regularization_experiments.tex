\section{Experiments with regularization techniques}

\subsection{Dropout layer on the input}

Denosing autoencoders\cite{denoising} improve the quality of the representations
by forcing resiliance of the neural network by adding noise on the input
and using the original datapoint in the objective function.

We tried a similar mechanism on our AutoRec re-implementation
by adding a Dropout \cite{Srivastava2014} layer with parameter $p=0.1$ on the input. 

The following plot shows that an improvement in generalization is indeed obtained:


\begin{figure}[H]
\centering
\includegraphics[scale=0.6]{autorec_input_dropout.png}
\caption{Dropout layer on the input of an AutoRec model}
% python3 plot_2.py  harvest_autorec_20180108_154116 harvest_autorec_20180120_124230 'without input dropout'  'input dropout p=0.1' --save text/autorec_input_dropout.png --epochs 627 --minerr 0.7 --maxerr 1.0
\label{input_dropout_fig}
\end{figure}
