\addchap{Preface}

When I was proposed me the topic
of collaborative filtering
I accepted with enthusiasm.
The idea of being able to infer user-on-item preferences
without any description of either users
or items fascinated me. How would it be
possible to do machine learning
having only relational information between different
entities? This less known application of machine learning still puzzles me and makes me
wonder about the incredible potential of these models.

This thesis has been a long journey with peaks and flats in which I could
experience both the excitement of attempting new 
ideas on how to solve the problem, as well as reconsidered expectations.
This is normal part of the life of any research scientist
and I'm glad of the opportunity of getting to know what this is all about.

The main aspect that motivates me into this thesis and in a broader scope to
Machine Learning and Artificial Intelligence is the sheer amount of new discoveries and
techniques that are being relentlessly being produced by the scientific community
and my desire to combine the state of the art in terms of neural models, 
update algorithms,
regularization techniques,
and probabilistic interpretations in order to push the boundary
of the best achievable precision of the predictions.
I'm always been interested in the nature of human conceptualization
and how this can be related to computability. This thesis
allowed me to get an additional perspective on this matter.

I believe that Collaborative Filtering techniques will find a broader application
that goes way beyond mere user/item rating prediction. They provide another way
to model learning by association, in which observations of how objects interact
lead to answers about what these objects actually are, also via interpretation
of their location in the so-called latent space.

I hope that the reader will find interesting how techniques that are typically
used for dimensionality reduction and probabilistic inference, with variational
approximations, have been employed for attempting a solution of
user/item rating modeling. I tried my best to derive all necessary math
in order to lead the reader to understand, step by step, the topics of
variational inference, the variational autoencoder and improvements to
the approximation such as the normalizing flows.

I also included a description of the experiments and their results of
many attempts at combining various algorithms
into searching
the combination that would lead to the best results
and some attempts at explaining different outcomes.
