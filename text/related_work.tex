\chapter{Related work}

\section{Probabilistic Matrix Factorization}
\emph{Probabilistic Matrix Factorization}\cite{salakhutdinov2008a}
is dimensionality-reduction technique for the CF problem
that learns a matrix factorization of the sparse matrix of
observed ratings \emph{R}
into two low-dimensional factor matrices
$U \in R^{D \times N}$ and $V \in R^{D\times M}$
where $D$ is the size of the low dimensionality.
Hence, $R = \transpose{U}V$.

The learning algorithm is based on a probabilistic assumption:
\begin{nalign}
p(R|U,V,\sigma^2) =
\prodiN
\prodjM
\left[
\normal{\Rij}{\Uit \Vj}{\sigma^2}
\right]^{\Iij}
\end{nalign}

Here $\Iij$ is $0$ if the $\Rij$ is not set and is $1$ if it is set.

In PMF, the log-likelihood is a sum of terms, each dependent on 
a specific user and item with $\Iij=1$.
This allows for SGD-like updates of the vectors $\Ui$ and $\Vj$
that are progressively refined trough the iterations.

\section{AutoRec}

\emph{AutoRec} \cite{Sedhain2015}, differently from
PMF, does not store learned latent vectors,
but is able to produce them on-the-fly via
an encoder-decoder neural network architecture.

This model is particularly interesting as a single query 
with an entire sparse ratings vector
results in all the missing ratings to be estimated at once.

The missing ratings are not provided to the encoder
but an estimation of those is nevertheless being provided
by the decoder, making use of the representation learning and denoising
intrinsic capabilities of an autoencoder with low dimensional bottleneck layer.

The loss function is hence the error
between the user (or item) vector $\boldr$ and its reconstruction, 
but considering, via element-wise 
multiplication $\odot$ with the vector mask $\mask$,
only the existing ratings, otherwise the learning would be incorrectly
taking account of the 0 placeholders for the missing ratings in the sparse matrix:

\begin{nalign}
\min \sum_{k} 
    \ltwonorm{
    \left[\boldrk - \Dec{\Enc{\boldrk}} \right]
    \odot
    \maskk
}
\end{nalign}

Even with this model, the sum allows for SGD-like updates.

\section{Matrix Factorizing Variational Autoencoder}
\emph{MFVA} \cite{vanBaalen2016} makes use of the findings in \cite{1312.6114}:
variational autoencoders are being used in order to yield
posterior distributions approximations as diagonal Gaussians 
$\qui$ and $\qvj$.

The decoder/recognition functions that have been used, differently from AutoRec,
output the single $\rij$ rating. A dot-product between $\boldui$ and $\boldvj$
as well as MLP have been employed for the task, with $\rij$ being
either expressed with a Gaussian distribution or with a multinomial distribution.

The lower bound assumes this form:

\begin{nalign}
\mathcal{L} &= -\sum_i\kl{Q_u(\boldui|r_{i,\cdot})}{p(\boldui)}\\
&-\sum_j\kl{Q_v(\boldvj|r_{\cdot,j})}{p(\boldvj)}\\
&+ \sum_{i,j}\expect{Q_u,Q_v}{\log p(r_{ij}|\boldui,\boldvj)}
\end{nalign}

