\section{Variational Lower Bound}

Given a 
dataset $\boldX = \{\boldxone \ldots \boldxN \}$
and the respective unobserved latent variable realizations
$\boldZ = \{\boldzone \ldots \boldzN\}$
\cite{Fox2012}
suggest to derive a quantity to be maximized by considering the minimization
of the Kullback-Leibler divergence $\kl{\qphiZ}{\pZcond}$ by decomposing it as follows:

\begin{nalign}
\kl{\qphiZ}{\pZcond} &=
\expectqphiZ{\log \frac{\qphiZ}{\pZcond}}\\
&= \expectqphiZ{\log \qphiZ - \log \pXcond - \log \pZ + \log \pX}
\end{nalign}

As the integrand inside the expectation $\expectqphiZ{\log \pX}$ 
is a constant w.r.t. $\boldZ$, then $\expectqphiZ{\log \pX}=\log \pX$. Hence, the
previous expression can be rewritten as:

\begin{nalign}
\log \pX &= \kl{\qphiZ}{\pXcond}
+ \elboX
\end{nalign}

Where we made use of this shorthand:
\begin{nalign}
    \elboX = \expectqphiZ{-\log \qphiZ + \log \pXcond + \log \pZ}
\end{nalign}

As $\log \pX$ is a constant w.r.t. the variational parameters $\phi$,
and $\kl{\qphiZ}{\pXcond}$ is always non-negative,
then the quantity $\elboX$ can be interpreted as a lower-bound to $\log \pX$
whose maximization implies the minimization of $\kl{\qphiZ}{\pXcond}$.

The lower-bound $\elboX$ can also be expressed, as in \cite{1312.6114},
by grouping some terms
into a negative Kullback-Leibler divergence:
\begin{nalign}
\elboX 
&= -\kl{\qphiZ}{\pZ} + \expectqphiZ{\log \pXcond}
\end{nalign}

More commonly, the lower-bound is re-arranged by making use of
the entropy of the variational approximation
and expectation over the joint probability:

\begin{nalign}
\elboX = \entropy{\qphiZ} + \expectqphiZ{\log \pXZ}
\end{nalign}



