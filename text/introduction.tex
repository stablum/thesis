\chapter{Introduction}

This work presents an exploration on the use of Variational Autoencoders for collaborative
filtering. The baseline model has been chosen to be \emph{AutoRec}\cite{Sedhain2015},
which uses latent representation to reconstruct missing ratings.
The natural evolution of this model has been considered to be similar
model based on a Variational-AutoEncoder, which we called \emph{VaeRec}.
Various extensions to this model have been examined.
Specifically, \emph{Normalizing Flows} transformations of the posterior approximation 
have been investigated, as well as regularization techniques.

The structure of this thesis represents how this research has evolved in time:
Chapter 2 offers a highlight of the notions required to delve into the 
actual contributions of the model ;
Chapter 3 presents similar models that have been used as inspiration ;
Chapter 4 presents our models with their variants ;
Chapter 5 illustrates the experiments that have been performed on our models ;
Chapter 6 summarizes the contributions of the models with insights that emerged from all the experiments ;
Most detailed derivations and proofs, very useful for a beginner that is trying to figure
out mathematical details of the models, have been left to the Appendix.

