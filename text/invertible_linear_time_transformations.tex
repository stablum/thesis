\section{Invertible Linear-time Transformations}

It has been proposed\cite{1505.05770} to use a type of transformations with the following form:

\begin{align}
t(\boldz) = \boldz + \boldu h(\wt \boldz + b)
\end{align}

$h$ is an element-wise nonlinearity. 
The parameters are: $b$ which is a scalar, 
$\wt \in \RD$ and $\boldu \in \RD$.

The expression $\wt \boldz + b$ is a scalar value,
and $h(\wt \boldz + b)$ can be seen as one perceptron layer
with a single output unit. 
$\boldu$ is a parameter that acts as a coefficient vector
representing the amount of the transformation $h(\wt \boldz + b)$
applied to the input $\boldz$ vector.

The previous derivations showed how just the 
determinant of Jacobian of the transformation
is used in order to express the probability of the transformed variable
as a function of the probability of the original variable $\boldzzero$.

The Jacobian $\partialboldz{t(\boldz)}$ of the transformation can be found as follows:

\begin{align*}
\partialboldz{\boldz} &= \identity \\
\partialboldz{\wt \boldz + b} &= \wt && \text{The Jacobian of a scalar is a row vector} \\
\partialboldz{h(\wt \boldz + b)} &= h^\prime (\wt \boldz + b)\wt && \text{Chain rule}
\end{align*}

Hence:
\begin{align}
\partialboldz{t(\boldz)} &= \identity + \boldu h^\prime(\wt \boldz + b)\wt
\end{align}

In order to derive the determinant of the Jacobian, it's possible to make use of the
\emph{matrix determinant lemma} which states:
\begin{align}
    \det ( \boldA + \bolda \boldbt ) = (1 + \boldbt \boldAinv \bolda)(\det \boldA)
\end{align}

Considering $\boldA = \identity$, $\bolda = \boldu$ and $\boldbt = h^\prime(\wt \boldz + b)\wt$,
this leads to:

\begin{align}
\det \partialboldz{t(\boldz)} &= \det(\identity + \boldu h^\prime(\wt \boldz + b) \wt)\\ 
&= (1 + h^\prime(\wt \boldz + b)\wt \identity^{-1} \boldu)(\det \identity)\\
 &=1 + h^\prime(\wt \boldz + b)\wt \boldu
\end{align}

