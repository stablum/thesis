\section{VAERec with Normalizing Flows}

The \emph{VAERec-NF} model
extends the VAERec by improving the posterior approximation
with Normalizing Flows \cite{1505.05770}
as explained in section \ref{iltt}, 
\ref{energy_of_1step_illt}
and \ref{multiple_iltt_steps}.

\subsection{Normalizing Flow using RealNVP's invertible transformation}

In this \emph{VAERec-NF} variant, a transformation of the type previously described
in section \ref{realnvp} is introduced.
This transformation was considered interesting 
because of its implementation
ease and very simple determinant of the Jacobian.
Specifically, the function $s$ is implemented as
a single perceptron layer with nonlinear activation function 
\emph{tanh}. The function $a$ which is required to be an affine transform,
is implemented as a single perceptron layer with \emph{linear}
activation function.

All the parameters of the transformation are being produced as output of the
encoder network, exactly as happens with the parameters of $\pzcond$ in a Variational
Auto-Encoder. This differs from the model of \ref{realnvp} as their network
parameters are not given as a function of the inputs, but is rather a
globally initialized global network which is the same for every input.
The weights of the $a$ and $s$ layers are implemented as vectors of size $K^2$, then
reshaped into $(K,K)$ dimensions. This limits the model into
very low latent dimensionalities.

\subsection{Masking}

To ease the implementation, the selection of the first and second parts of $\boldz$
have been implemented with random hyperparameter masks. These masks are unique for each transformation
step, and are computed as follows:

\begin{algorithm}[H]
\caption{Half-full random masks for RealNVP transformations}
\begin{algorithmic}[1]

\REQUIRE ~~\\
(1) Latent dimensionality $K$ \\
(2) Number of transformation steps $k$
\ENSURE~~\\
(1) Random masks $\boldm_1 \ldots \boldm_k$ which have half of their elements set at $1$
`
\item[]
\FOR{$i \in \{1 \ldots k\}$}
\STATE $(\mathbf{a})_j \leftarrow \left\{\begin{array}{ll} 1 & j < K/2 \\ 0 & K/2 \leq i < K\end{array}\right.$
\STATE $\boldm_i = \mathrm{shuffle}(\mathbf{a})$
\ENDFOR
\RETURN $\boldm_1 \ldots \boldm_k$
\end{algorithmic}
\end{algorithm}
                                         
The invertible function, for a transformation step $i$, is hence implemented as:

\begin{nalign}
    \boldz_{i+1} &= \boldz_i\odot\boldm_i + (1-\boldm_i)\odot\left[\boldz_i \odot \exp\left(s(\boldm_i\odot\boldz)\right) + a(\boldm_i\odot\boldz)\right]
\end{nalign}

The masks are computed before training and are left unchanged, as they
should be considered as hyperparameters.
