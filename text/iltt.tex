\section{Planar Transformations}\label{iltt}

It has been proposed\cite{1505.05770} to achive a more complex 
posterior approximation
by using a type of transformations with the following form:

\begin{nalign}
t(\boldz) = \boldz + \boldu h(\wt \boldz + b)
\end{nalign}

This transformation can be applied to a simpler distribution,
such as the diagonal-covariance gaussian
introduced in \cite{1312.6114}.

The parameters are: $b$ which is a scalar, 
$\wt \in \RD$ and $\boldu \in \RD$;
$h$ is an element-wise nonlinearity,
such as a sigmoid $h(a) = \frac{1}{1+e^{-a}}$.


The expression $\wt \boldz + b$ is a scalar value,
and $h(\wt \boldz + b)$ can be seen as one perceptron layer
with a single output unit. 
$\boldu$ is a parameter that acts as a coefficient vector
representing the amount of the transformation $h(\wt \boldz + b)$
applied to the input $\boldz$ vector.

The derivations in Appendix \ref{density_transformed} show how just the 
determinant of Jacobian of the transformation
is used in order to express the probability of the transformed variable
as a function of the probability of the original variable $\boldzzero$.
For the derivation of the Jacobian please refer to Appendix \ref{jacobian_illt}
